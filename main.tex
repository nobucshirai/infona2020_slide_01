\documentclass[dvipdfmx,aspectratio=169,20pt]{beamer}
\usepackage{bxdpx-beamer}

% Beamer theme
\usetheme{Boadilla}

%%%%% JAPANESE FONT SETTINGS %%%%%
%\usepackage{minijs}
\usepackage[utf8]{inputenc}
\usepackage{pxjahyper}
\renewcommand{\kanjifamilydefault}{\gtdefault} % for Gothic Japanese fonts
\usepackage[deluxe,uplatex]{otf} % needed to use super bold Japanese fonts
\usepackage[unicode,noto-otc]{pxchfon} % needed to use super bold Japanese fonts
%%%%%%%%%%%%%%%%%%%%%%%%%%%%%%%%%%

%%%%% SETTINGS FOR MATH SYMBOLS %%%%%
\usepackage{amsmath,amssymb}
\usepackage{bm}
\usepackage{latexsym}
\usefonttheme{professionalfonts} % use Serif fonts for equations
%%%%%%%%%%%%%%%%%%%%%%%%%%%%%%%%%%%%%

%%%%% GRAPHIC SETTINGS %%%%%
%\usepackage[dvipdfmx]{graphicx}
%\usepackage{wrapfig}
%%%%%%%%%%%%%%%%%%%%%%%%%%%%


\usepackage{fancybox,ascmac}
\usepackage{url}
\usepackage[many]{tcolorbox}

%%%%% ALGORITHM SETTING %%%%%
\usepackage{algorithm}
\usepackage[noend]{algorithmic}
\algsetup{linenosize=\color{fg!50}\footnotesize}
\renewcommand\algorithmicdo{\bfseries :}
\renewcommand\algorithmicthen{\bfseries :}
\renewcommand\algorithmicrequire{\textbf{Input:}}
\renewcommand\algorithmicensure{\textbf{Output:}}
%%%%%%%%%%%%%%%%%%%%%%%%%%%%%

%% tikz関係 %%%
\usepackage{tikz}
\usetikzlibrary{matrix}
\usetikzlibrary{positioning}
\tikzset{>=stealth}
\usetikzlibrary{angles,arrows.meta,backgrounds,calc,intersections,positioning,quotes,scopes,shapes.arrows,through}
\usepackage[customcolors]{hf-tikz}
\tikzset{offset def/.style={above left offset={-0.1,0.8},below right offset={0.1,-0.65},},integral first/.style={offset def,},integral second/.style={offset def,set fill color=green!50!lime!60,set border color=green!40!black,},sums/.style={offset def,set fill color=blue!20!cyan!60,set border color=blue!60!cyan,}}
\setbeamertemplate{navigation symbols}{}
%%%%%%%%%%%%%%%%%%%%%%%%%%%%%%%%%%%%%%%%%%%%%%%%
%\definecolor{myblue1}{RGB}{10,30,190}
\definecolor{myblue1}{RGB}{45,130,200}
%\definecolor{myblue1}{RGB}{35,119,189}
\definecolor{myblue2}{RGB}{95,179,238}
\definecolor{myblue3}{RGB}{129,168,207}
\definecolor{myblue4}{RGB}{26,89,142}

\setbeamercolor*{structure}{fg=myblue1,bg=blue}
\setbeamercolor{block title}{fg=myblue1!50!black}
\setbeamercolor*{block title example}{fg=white,bg=myblue4}
\setbeamercolor*{palette primary}{use=structure,fg=white,bg=structure.fg}
\setbeamercolor*{palette secondary}{use=structure,fg=white,bg=structure.fg!75!black}
\setbeamercolor*{palette tertiary}{use=structure,fg=white,bg=structure.fg!50!black}
\setbeamercolor*{palette quaternary}{fg=black,bg=myblue1}

\setbeamerfont{alerted text}{series=\bfseries}
\setbeamerfont{section in toc}{series=\mdseries}
\setbeamerfont{frametitle}{size=\Large,series=\bfseries}
\setbeamerfont{title}{size=\LARGE,series=\bfseries}
\setbeamerfont{date}{size=\small}

\setbeamertemplate{block title}[shadow=false]
\setbeamertemplate{blocks}[rounded][shadow=false]

%\makeatletter

%%%%% BEAMER FOOTLINE SETTINGS %%%%%%
\setbeamertemplate{footline}[frame number]{}
\setbeamerfont{footline}{size=\bf\footnotesize\small}
%%%%%%%%%%%%%%%%%%%%%%%%%%%%%%%%%%%%%

%%%%% BEAMER ITEM SETTINGS %%%%%
\setbeamertemplate{itemize item}[circle]
\setbeamertemplate{itemize subitem}[triangle]
\setbeamertemplate{enumerate item}[circle]
%%%%%%%%%%%%%%%%%%%%%%%%%%%%%%%%

\newcommand{\firstreportid}[0]{I}
\newcommand{\secondreportid}[0]{I\hspace{-.01em}I}

    %タイトルページ
\setbeamertemplate{title page}{%
    \vspace{1.2cm}
    \hspace{1.5cm}{\selectfont\usebeamerfont{subtitle} \usebeamercolor[fg]{subtitle} [数値解析 第1回] \par}
    \vspace{0.5cm}
    %\vspace{2.5em}
    {\centering\usebeamerfont{title} \usebeamercolor[fg]{title} \inserttitle \par}
    \vspace{0.5cm}
    \begin{center}
        丸めた数値と真面目に向き合う
    \end{center}
}
%%%%%%%%%%%%%%%%%%%%%%%%%%%%%%%%%%%%%%%%%%%%%%%%

%Information to be included in the title page:
\title{有限桁の呪縛}

\date[\todey]{}

\begin{document}

\frame{\titlepage}

%%%%%%%%%%%%%%%%%%%%%%%%%%%%%%%%
\begin{frame}
\frametitle{数値解析とは}
\begin{block}{{\bf 数値解析} {\small (Numerical Analysis)}}
数学の問題を数値を用いて近似的に解く応用数学の一分野
\end{block}
\begin{itemize}
    \item 現代的には「数値を用いて」の部分を{\bf 「計算機 (コンピュータ) を用いて」}としても良い
\end{itemize}
\end{frame}
%%%%%%%%%%%%%%%%%%%%%%%%%%%%%%%%
\begin{frame}
\frametitle{{\large 数値解析に現れる2つの有限}}
\begin{itemize}
    \setlength{\itemsep}{0.5cm}
    \item 数値解析は{\bf 2つの有限縛り}を数学に加えたもの
    \item {\bf 有限桁縛り}
    \begin{itemize}
        \item 数値は有限桁までしか扱えないという縛り
    \end{itemize}
    \item {\bf 有限回縛り}
    \begin{itemize}
        \item 繰り返しの処理は有限回までしか試行できないという縛り
    \end{itemize}
\end{itemize}
\end{frame}
%%%%%%%%%%%%%%%%%%%%%%%%%%%%%%%%
\begin{frame}
\frametitle{{\large 数値解析で目指すもの}}
\begin{itemize}
    \setlength{\itemsep}{0.5cm}
    \item {\bf 速さ}
    \begin{itemize}
        \item 同じ問題をより速く解ける
    \end{itemize}
    \item {\bf 安さ}
    \begin{itemize}
        \item 記憶領域を節約できる
    \end{itemize}
    \item {\bf 旨さ}
    \begin{itemize}
        \item 近似解の質が高い (誤差が小さい)
    \end{itemize}
\end{itemize}
\end{frame}
%%%%%%%%%%%%%%%%%%%%%%%%%%%%%%%%
\begin{frame}
\frametitle{[問題] \firstreportid-A}
有効数字10進3桁で表された数値 $a=0.537$, $b=0.612$, $c=0.126$ に対して $(a+b)+c$ と $a+(b+c)$ を計算せよ。ただし加算の際に得られた数値は有効数字10進4桁以降を切り捨てて計算せよ。
\end{frame}
%%%%%%%%%%%%%%%%%%%%%%%%%%%%%%%%
\begin{frame}
\frametitle{[略解] \firstreportid-A}
\vspace{-1cm}
\begin{eqnarray*}
    (a&+&b)+c\\
    &=&(0.537+0.612)+0.126\\
    &=&1.14+0.126=1.26
\end{eqnarray*}
\begin{eqnarray*}
    a&+&(b+c)\\
    &=&0.537+(0.612+0.126)\\
    &=&0.537+0.738=1.27
\end{eqnarray*}
\end{frame}
%%%%%%%%%%%%%%%%%%%%%%%%%%%%%%%%
\begin{frame}
\frametitle{{\large [解説] 有限桁の呪縛}}
\begin{itemize}
    \setlength{\itemsep}{0.5cm}
    \item {\bf 結合法則が成り立たない}
    \begin{itemize}
        \item $(a+b)+c\neq a+(b+c)$
    \end{itemize}
    \item 有限桁の数値の加算は{\bf 足す順番に依存する}
    \begin{itemize}
        \item 演算の度に丸め操作を繰り返すと{\bf 丸め誤差}が蓄積するため
    \end{itemize}
\end{itemize}
\end{frame}
%%%%%%%%%%%%%%%%%%%%%%%%%%%%%%%%
\begin{frame}
\frametitle{{\large [用語解説]}}
\begin{block}{{\bf 丸め誤差} {\small (rounding error)}}
ある数値を有限桁に丸めることによって生じる誤差
\end{block}
\begin{itemize}
    %\setlength{\itemsep}{0.5cm}
    \item 丸め操作
    \begin{itemize}
        \item 問題\firstreportid-A では「10進4桁以降を切り捨て」の部分が丸めの操作
    \end{itemize}
    \item 有限桁しか扱えない数値解析では丸め誤差がいつもつきまとう
\end{itemize}
\end{frame}
%%%%%%%%%%%%%%%%%%%%%%%%%%%%%%%%
\begin{frame}{{\large [手法] 旨くするための工夫}}
    \begin{block}{{\bf\small カハンの保証付き総和法}\\
    \vspace{-5mm}
    {\fontsize{13pt}{18pt}\selectfont (Kahan summation algorithm/compensated summation)}}
        {\fontsize{15pt}{18pt}\selectfont
        \begin{algorithmic}[1]
            \REQUIRE $a_1,\dots,a_n,n$
            \ENSURE $sum$
            \STATE $sum \leftarrow 0.0$; $cp \leftarrow 0.0$
            \FOR{$i = 1,2, \dots, n$}
            \STATE $y \leftarrow a_i - cp$
            \STATE $t \leftarrow sum + y$
            \STATE $cp \leftarrow (t-sum) - y$
            \STATE $sum \leftarrow t$
            \ENDFOR
            \RETURN $sum$
        \end{algorithmic}
        }
    \end{block}
\end{frame}
%%%%%%%%%%%%%%%%%%%%%%%%%%%%%%%%
\begin{frame}
\frametitle{[問題] \firstreportid-B}
有効数字10進3桁で表された数値 $a=6.52$, $b=6.57$ に対して $\frac{a+b}{2}$ と $a+\frac{b-a}{2}$ を計算せよ。ただし加減乗除の際は得られた数値の有効数字10進4桁以降を切り捨てて計算せよ。
\end{frame}
%%%%%%%%%%%%%%%%%%%%%%%%%%%%%%%%
\end{document}